\chapter{Data}

Because human nerve data is hard to obtain, we can also use animal data instead as a proxy. Animal data is usable as proxy because we can observe the nerve fibers in vitro but can better separate one single nerve fiber from others. In human data an additional step of fiber separation is necessary to differentiate between individual fibers. We can use the same experimental protocols on Animals as we would on humans. This way we can understand firing patterns of spikes and quantify them. The results can then be applied to human data. \\
In the case of this thesis, we are using the data from wistar rats. The data was recorded from 2011 to 2012 by Roberto de Col and was published in a paper (put reference). The goal of the paper was to evaluate the effects of spiking activity on the response to mechanical stimulation. 

-How much of the exact experimental details are supposed to go here as far as methodology goes, since this is a computer science thesis 

The experiments were done in vitro on peripheral nerve fibers. The fibers were mechanically and electrically stimulated via a custom made electromechanostimulator. The nerve activity was recorded using an electrode. The electrical stimulation consists of small electrical pulses that come in a controlled frequency. The mechanical force is applied in a sinusoidal shape. \\
For single recordings the mechanical force that is applied throughout stays at approximately the same level for most of the files (put for how many files this is the case), but there are exceptions where the mechanical stimulation changes in amplitude and length during one recording. 

The experimental software used for these experiments is called Spike2 and is described further in the background chapter.

\cleardoublepage
