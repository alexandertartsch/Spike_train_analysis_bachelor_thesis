\chapter{Conclusion}
In this thesis, spike train analysis in mechanically and electrically stimulated rat neural fibers was studied. This analysis was tested as a use case for the openMNGlab software. As a proxy for human MNG data single nerve fiber recordings from the skin nerve preparation of rats was used. Regular electrical stimulation was applied, to study the effect of previous nerve fiber activity on the response to mechanical stimuli. 
The result of this was that the more the nerve fiber was active, the less was the response to mechanical stimuli. After a threshold of 2 Hz of electrical stimulation frequency the mechanically evoked spike trains contain noticeably less spikes and the peak firing frequency is significantly lower.

The openMNGlab framework was used to facilitate the analysis. The analysis of this type of mechanically stimulated data showed that the importing capabilities of the software needs improvements. Not all of the data could be loaded with the standard tools included in the framework. Other users also reported issues with the importers of openMNGlab, which is where the main work in the near future lies. The code used in this bachelor thesis can be found under \url{https://github.com/alexandertartsch/Bachelor_thesis_code/tree/v1}

% Bild einbinden
%\begin{figure}[ht!]
%\begin{center}\includegraphics[width=5cm]{src/pic/logo}\end{center}
%\caption{Das SE Logo}
%\label{Logo}
%\end{figure}

\cleardoublepage
