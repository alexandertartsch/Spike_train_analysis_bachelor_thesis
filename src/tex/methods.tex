\chapter{Methods}
%not sure if use-cases should not go into a different chapter\\
%this chapter would then focus on just the process of developing my analysis pipeline and finish with a detailed description of it.
%\section{Chapter content}
\begin{comment}
In this chapter I will describe the methods I used for creating the software and spike analysis results in this thesis\\
-Describe the different data structures (Neo, old openMNGlab version)\\
-Detail the development process, issues during development and how they were resolved\\
-Describe the finished analysis pipeline with the help of a simplified graph\\
-present use cases (3 students analysing different data, experimental scientist)\\

Spike Analysis:\\
-Data\\
-definition of spike trains\\
-quantifiers\\

TODO:\\
-Add what I did: somewhat detailed description of my process in developing the notebook\\
-start from the notebook which was started by Radomir for custom structure (before openMNGlab)\\
-adding of importers from old openMNGlab version (code from Fabian)\\
-adding of neo importers after integration into openMNGlab\\
\end{comment}
\section{Software}
The methods used in this thesis are twofold. On the one hand I tried to analyze the recordings of mechanically and electrically stimulated rats using the same quantifiers as in the reference paper~\cite{roberto}. On the other hand I wanted to get an overview over the state of the openMNGlab framework~\cite{schlebusch_openmnglab_2021}. My goal was not to provide a fully functioning analysis system as part of this thesis, but rather comment on the current issues with the software. For this I also talked to other users of the framework to get a broader view of the system.
I have split this chapter in two parts that deal with the Software aspect and Analysis aspect of this thesis respectively.

The first step in the process to understanding the problems with the software was to use the software myself. By using the software I could test for potential issues while also doing work towards the analysis.
While using the software I encountered various different errors and issues of varying size and importance.
After using the software for myself and getting to know the framework for my own specific use case I also interviewed other users about the framework to get an idea of other potential issues with the software that I was not aware of during my time working with it.
In the end I received a comprehensive list of problems with the software and gained a good understanding of various users needs. From this point of view I can then propose solutions and ideas for error fixing.


\section{Spike Analysis}
In the second part of this chapter I will now go into the methodology used for the concrete analysis, as well as describe some of the basic concepts that we will be needing further.

\subsection{Data}
%This subsection will contain a description of the original experiment data from Roberto de Col.
An integral part of the analysis is of course the data itself. Because human nerve data is hard to obtain, we can also use animal data instead as a proxy. Animal data is usable as proxy because we can observe the nerve fibers in vitro but can better separate one single nerve fiber from others. In human data an additional step of fiber separation is necessary to differentiate between individual fibers. We can use the same experimental protocols on Animals as we would on humans. This way we can understand firing patterns of spikes and quantify them. The results can then be applied to human data. \\
In the case of this thesis, we are using the data from wistar rats. The data was recorded from 2011 to 2012 by Roberto de Col and was published in a paper~\cite{roberto}. The goal of the paper was to evaluate the effects of spiking activity on the response to mechanical stimulation. \\
%-How much of the exact experimental details are supposed to go here as far as methodology goes, since this is a computer science thesis 
The experiments were done in vitro on peripheral nerve fibers. The fibers were mechanically and electrically stimulated via a custom made electromechanostimulator. The nerve activity was recorded using an electrode. The electrical stimulation consists of small electrical pulses that come in a controlled frequency. The mechanical force is applied in a sinusoidal shape. \\
For single recordings the mechanical force that is applied throughout stays at approximately the same level for most of the files, but there are exceptions where the mechanical stimulation changes in amplitude and length during one recording. 

\subsection{Spike Train Definition}
Earlier in this thesis we covered the basics of electrical signals in nerve fibers and action potentials, or spikes. In this thesis we are, however, not interested in single spikes but in clusters of them. For this it is important to define what we mean when we speak of spike trains.\\
In theory any number of spikes could be called a spike train. What we are interested in is the short amount of time after a stimulus event where there is increased spiking activity, see figure~\ref{fig:spike_train}. In the data provided by Roberto de Col there is mechanical and electrical stimulation. The mechanical stimulation, which is approximately 250 milliseconds long, leads to noticeably increased spiking activity. It is this spiking activity that we want to analyze and quantify and to have a unified way of speaking about this activity we call these clusters of spikes spike trains. We consider the mechanical stimulus the start of the spike train. This is marked in the event channel of the data. A spike train is considered finished after the increased spiking activity has normalized. This can have a duration ranging from 150 milliseconds to 500 milliseconds. Due to computational reasons in our spike train detection we are not using any complicated algorithms but only search an interval of 500 milliseconds after the mechanical stimulus to look for spikes for the spike trains. Any spikes happening after that interval do not get counted towards the spike train with our current method. All of the spikes appearing inside of this interval get counted as belonging to the spike train.

\subsection{Spike Train Analysis: Quantifiers}
Now that we had a look at the data that we have available, it is time to discuss the quantifiers that I chose to use for the analysis of the data.
Starting out with raw data the first and perhaps most obvious quantifier that can be computed is the number of spikes in a spike train. It is a basic measure and can give information at a quick glance about the spike train. Of course this is not as expressive as other quantifiers and does not necessarily provide us with much context, but it can give hints at less or more active nerve fibers for example. \\
Another quite simple quantifier I found in the literature is the mean firing rate. This is calculated by dividing the duration of the spike train by the number of spikes and gives an idea of the frequency in which the nerve fiber fires at this moment.\\
The next quantifier I chose to include is the peak firing frequency. The reason for the inclusion is because it is one of the quantifiers mentioned in Roberto de Cols original paper about this data set. The peak firing frequency is calculated for each spike train and details the concentration of spiking activity. During development I experimented with two different implementations of calculating the peak firing frequency. First I tried sorting the instantaneous frequencies of a spike train by value and averaging over the 5 highest frequencies. This, however, does not give a representative look on a single section of spikes. I therefore switched to a sliding window implementation, where I calculated the average instantaneous frequency over a sliding window of width 5 and chose the highest average as the peak firing frequency.\\
\begin{comment}
At the request of my advisor I calculated the inter spike intervals(ISI) for the spike trains. This can give us a measure of the distribution of spikes in a spike train.\\
Going off of the ISI we can compute the linearity of the inter spike intervals in the form of R2. We want to know if the ISIs tend to increase as the spike trains last longer or decrease or if there is no significant impact. From the start we suspected that the intervals increase as the spike train goes on and calculating the R2 value should give us the confirmation.
\end{comment}



\cleardoublepage
