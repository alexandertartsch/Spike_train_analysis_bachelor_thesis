\vspace*{2cm}
% Abstract
%{\bf\Large Kurzfassung} \\ [1em] 
%Eine kurze Zusammenfassung der Arbeit.

%\vspace{10ex}
{\bf\Large Abstract} \\ [1em]
In medical sciences the study of pain is an important field. Pain fulfills an important warning function in the body. This can be disturbed, when the nervous system itself gets affected by illness. Important research is done in order to understand the neural signalling of pain inside the nervous system.

In this thesis we analyze spike trains in mechanically and electrically stimulated rat neural fibers. The goal of this analysis is to find out how previous nerve fiber activity effects the response of a fiber to mechanical stimuli. Eventually this is supposed to help in the field of neuropathic pain study. In this thesis we confirm the results of a paper by Uebner et al. A previously active nerve fiber has a lessened response to mechanical stimuli.

To facilitate the analysis we make use of the openMNGlab framework to test how this use case functions with the software. OpenMNGlab is a software tool that is specialized for microneurography data. It is supposed to import data from different sources, bring the data in a unified format and provide analysis capabilities. In addition to the use case of mechanically stimulated data, other users were also interviewed and their experiences are detailed in this thesis. With the ideas brought forward in this thesis, we hope to offer possible improvements to openMNGlab so that it can become a useful and important tool in the field of microneurography.

The code produced for and used in this bachelor thesis can be found under \url{https://github.com/alexandertartsch/Bachelor_thesis_code/tree/v1}

\cleardoublepage
