\chapter{Discussion}

In this chapter I will discuss the results presented in the previous chapter and think about possible future work regarding the analysis. Also I will discuss the integration of my code (quantifiers mainly) into openMNGlab and thoughts about the structuring of the software.

There seems to be a threshold of electrical stimulation frequency after which we notice a change in response to the mechanical stimuli. In the diagrams where we compare different quantifiers we see that a electrical background stimulation frequency of up to $2$ Hertz does not seem to have any effect on the peak firing frequency or the number of spikes per train. With the $4$ and $5$ Hertz stimulation, however, we see a noticeable dip in peak firing frequency and number of spikes per train. This would suggest that a high load on the nerve fiber hinders the regular response to the mechanical stimulus that we see without much activity on the nerve fiber.

discuss the software engineering of openmnglab

\section{Future Work}
Since we did not use sophisticated algorithms to detect spike trains, this would be a topic for further research. Our current method relies heavily on mechanical stimulus events to tell us where we can expect a spike train to appear. This is a luxury that is not a given in microneurography data and as such it is a problem worth looking into.\\
Another interesting possibility for the future is the incorporation of advanced analysis functions from the likes of Elephant into OpenMNGlab. When the structure allows for it it could yield a nice benefit to the analysis capabilities of researchers who are using OpenMNGlab.\\
In this thesis we focused mostly on the analysis of single recordings from mechanically and electrically stimulated rats. This could be expanded to more big picture analysis comparing multiple different recordings in a broader context than in this thesis, as well as comparing differing stimulation types. Other students at the chair for medical informatics are already working on recordings with other stimulation types for example.\\





\cleardoublepage
