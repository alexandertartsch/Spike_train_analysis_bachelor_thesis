\chapter{Discussion}
\begin{comment}
-speak about the problems with the software
- what needs to be fixed


-analysis:
-threshold, already mentioned in Robertos paper
-active nerve fiber has decreased response
-not linear

hdf5 not yet feasible, since we get 3 separate data structures and not one single concise format.


In this chapter I will discuss the results presented in the previous chapter and think about possible future work regarding the analysis. Also I will discuss the integration of my code (quantifiers mainly) into openMNGlab and thoughts about the structuring of the software.
\end{comment}
\section{Software}
This thesis included working with multiple versions of the openMNGlab framework. The biggest issue with the software at this point are the importing capabilities. This is a very crucial functionality necessary for everyone who works with the software. Different users have reported that not all aspects are imported into the framework, such as the "Comments" channel containing the experimental protocol in chemical recordings. 

There are other cases where the functionality can be improved. In this thesis, there were difficulties importing everything that was needed for the analysis. With the current Neo importer and mechanically and electrically stimulated data, there is mechanical information that gets lost, when just using the current framework version. This is something that has to be rectified and the import of the mechanical force channel, as well as the missing wavemark channels might be possible in the future. 

Another topic worth discussing, is the possibility of storing the data and having the option of reloading it at a later time. This would lead to faster loading times, when revisiting already analyzed data. Neo supports this by providing the option to save Neo objects to the hdf5 file format. This could be used for the thesis data only if everything can be imported by the Neo importer and put into one single data structure, which is another reason to look into this in the future. 

Since every experiment produces different data with different specifications, there might come a point in the future, where new issues with new data sets arise.

\section{Spike Analysis}
In Chapter~\ref{results_chapter}, we have seen what effect continuous nerve fiber activity can have on the response to a new stimulus. With increased fiber activity, induced by electrical stimulation, the spike train elicited by a mechanical stimulus changed in comparison to the control. The higher the electrical stimulation frequency gets, and with it the fiber activity, the fewer spikes are produced by the mechanical stimulus. This relation is not linear, however. There seems to be a threshold of electrical stimulation frequency after which we notice a change in response to the mechanical stimuli. In Figures~\ref{fig:quantcomp_sp},~\ref{fig:inst_freqs} we see that an electrical background stimulation frequency of up to 2 Hz does not seem to have any effect on the peak firing frequency or the number of spikes per train. With the 4 and 5 Hz stimulation, however, we see a noticeable dip in peak firing frequency and number of spikes per train. This would suggest that a high load on the nerve fiber hinders the regular response to the mechanical stimulus that we see without much activity on the nerve fiber. This corresponds well with the results found by Uebner et al.~\cite{roberto}.

In this thesis, the detection of spike trains is very basic. We get the start time by the event marker in the recording file and then define the spike train as all the spikes that appear in a period of up to 500 ms after that event. For the purposes of these experiments this is enough, since we will always know when the spike trains are supposed to happen. When this sort of analysis gets expanded to also search for randomly appearing spike trains in patients in the future, there needs to be more sophisticated ways to detect spike trains when there is no way to know exactly when one will happen.

Another interesting aspect for the future is the integration of Elephant into the existing analysis. Since Elephant is also a Python module and implements advanced analysis functions for spike trains it might serve as a good addition. 

In this thesis, we focused mostly on the analysis of single recordings from mechanically and electrically stimulated rats. This could be expanded to more big picture statistics comparing multiple different recordings in a broader context than in this thesis, as well as comparing differing stimulation types. These statistics could give a better understanding of nerve fiber responses as a whole.





\cleardoublepage
