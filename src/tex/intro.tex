\chapter{Introduction}
\section{Chapter content}
In this chapter I will give an introduction to the topic. I will give background information about neuropathic pain and the big picture goal of research in this field. \\
-quickly explain how the transmission of pain functions inside our bodies\\
-what is my bachelor thesis based on (paper from Roberto)\\
-contextualize thesis topic within the big picture \\
-talk quickly about openMNGlab and goal of adding analysis functionality and discuss software engineering goals for the framework\\


% Die Logos sind veraltet und duerfen zurzeit nicht verwendet werden!
% Auf Seite \pageref{Logo} in Abbildung \ref{Logo} befindet sich das SE Logo.
\begin{comment}
-Neuropathic pain as basis \\
-comes with many diseases \\ 

-pain as electric signals \\
-goal to understand the firing patterns in nerve fibers \\
-microneurography as recording technique \\
-needle in vitro in patients \\
-action potentials as spikes \\
-animal data \\
-does not need fiber separation  

-OpenMNGlab \\
-currently only good for loading the data \\
-want to add analysis capabilities \\
-compute quantifiers for spike trains and recordings \\
-discuss results 
 
Software: \\
-working on jupyter notebook \\
-automate the spike analysis process \\
-integrate analysis into openMNGlab \\
-get requirements for spike analysis software 


There are many unsolved problems in the medical sciences. One of which is neuropathic pain that often occurs as a side effect of other diseases such as Diabetes for example. \\
-TODO: Do more research on neuropathic pain \\
Pain is transmitted as electric signals through the nerves inside our bodies. Because of this fact in order to understand different types of pain such as neuropathic pain for one we need to understand the transmission of electrical signals in nerve fibers. \\
Measuring electrical signals can be done in different ways. One of which is called microneurography. This is a technique that is used to record nerve activity in peripheral nerves. With this technique typically a needle gets inserted into a nerve fiber which then detects the electrical current in the fiber. Additionally, we can stimulate the nerve fiber to get certain responses. \\
Nerve fibers transmit data with the use of action potentials, short AP or spike. It has been shown in previous research that information is not transmitted by the shape or amplitude of the spikes, but the frequency and timings (look the exact papers up again). 

\cite{SE10}
\end{comment}
\cleardoublepage
